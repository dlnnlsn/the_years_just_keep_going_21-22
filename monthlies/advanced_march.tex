\documentclass{article}

\usepackage{mathtools,amsfonts}
\usepackage{enumitem}
\usepackage{fullpage}
\usepackage{fancyvrb}
\usepackage{hyperref}

\newcommand*{\sptext}[1]{\ \text{#1}\ }

\begin{document}
\thispagestyle{empty}

\begin{center}
  \textbf{\Large Advanced March Problem Set}
  \\ \vspace{1em}
  \textbf{\large Due: 31 March 2022}
\end{center}

\smallskip

\begin{enumerate}[itemsep=\fill]

\item % Finland 2009
For which $z \in \mathbb{Z}$ does there exist a polynomial $P$ with integer coefficients such that $P(2021) = 2022$ and $P(z) = z$?


\item % Centroamerican 2018
Liam and Dylan play a game, alternating turns with Liam playing first.
At the start, there is a pile of $1+2+\dotsb+2022$ stones.
On his turn, a player chooses a positive integer $n$ between 1 and 2022 inclusive such that neither player has previously chosen this $n$.
He then removes $n$ stones from the pile and keeps them next to him.
Once the pile is empty, the winner is the player with an even number of stones next to him.
Who has the winning strategy? 


\item % LB-2016-2
Let $ABC$ be a triangle.
Let $M$, $N$, and $P$ be points on sides $BC$, $CA$, and $AB$ respectively.
Let $BN$ and $CP$ intersect at $D$, $CP$ and $AM$ intersect at $E$, and $AM$ and $BN$ intersect at $F$.
Let $H$ be the centre of the circle passing through $D$, $E$, and $F$.
Prove that $AH \perp BC$.


\item % Dylan?
Consider the infinite sequence that starts with the six terms $1, 0, 1, 0, 1, 0$. From the seventh term onwards, each term is the units digit of the sum of the previous six terms. Show that the six numbers $0, 1, 0, 1, 0, 1$ never appear consecutively in the sequence. (In that order.)


\item % JM-2014-1
Let $\mathbb{N}$ denote the set of positive integers. Consider the function $f: \mathbb{N} \times \mathbb{N} \to \{-1,1\}$, defined as follows: 
\[
  f(i,j) = \begin{cases}
    \mspace{72mu} -1 & \sptext{if} i=1 \sptext{or} j=1 \\
    f(i-1,j) \cdot f(i,j-1) & \sptext{if} i>1 \sptext{and} j>1.
  \end{cases}
\]
Determine the largest $i \leq 2021$ such that $f(i,2021) = -1$.


\item % Russian MO final round, problem 11.4
Phil the Magician and Malwande his assistant have a deck of cards; the front side of each card is one of $2022$ non-white colours, and there are $1000000000$ cards of each non-white colour in the deck, but the back sides of all the cards are white.
They want to perform the following trick in front of an audience:
\begin{itemize}
  \item Phil the Magician leaves the room.
  \item The audience puts $n$ cards in a row onto a table, with the front sides facing up.
  \item Malwande his assistant looks at these cards, chooses one of them, and turns all the other cards face down (he does not change the order of the cards).
  \item Phil the Magician comes back into the room, looks at the table, chooses one of the face down cards and correctly announces its colour.
\end{itemize}
Find the least value of $n$ such that Phil the Magician and Malwande his assistant can figure out a strategy beforehand to always successfully perform this trick.

\end{enumerate}


\vfill
\footnotesize
\begin{itemize}
	\item Submit your solutions at \href{https://forms.gle/Pv89v957obJMEAw26}{https://forms.gle/Pv89v957obJMEAw26}
	\item Submit each question in a single separate PDF file (with multiple pages if necessary).
	\item If you take photographs of your work, use a document scanner such as Office Lens to convert to PDF.
	\item If you have multiple PDF files for a question, combine them using software such as PDFsam.
\end{itemize}

% \vfill
% ASCII art
\centering
\footnotesize
\begin{BVerbatim}
       /`-._
      /_,.._`:-         
  ,.-'  ,   `-:..-')   
 : o ):';      _  {   
  `-._ `'__,.-'\`-.)
      `\\  \,.-'`
\end{BVerbatim}

\end{document}
