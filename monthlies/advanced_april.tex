\documentclass{article}
\usepackage{mathtools,amsfonts}
\usepackage{enumitem}
\usepackage{fullpage}
\usepackage{fancyvrb}
\usepackage{hyperref}


\begin{document}
\thispagestyle{empty}

\begin{center}
  \textbf{\Large Advanced April Problem Set}
  \\ \vspace{1em}
  \textbf{\large Due: 30 April 2022}
\end{center}

\bigskip

\begin{enumerate}[itemsep=12pt]

\item %Czech and Slovak MO 2016
Find the smallest positive integer that can be inserted between numbers $20$ and $16$ so that the resulting number $20 \dots 16$ is a multiple of $2016$.


\item % DB-2016-2
For which real numbers $s$ and $t$ does the polynomial $p(x) = X^3 -(2s+t)X^2 +(2s^2+t^2)X +3st$ have three (not necessarily distinct) real roots?


\item % Spain 2019 Q2
Is there a finite set $P$ of prime numbers such that for every integer $n \geq 2$ the number $2^2 +3^2 +\dotsb +n^2$ is divisible by at least one of the elements of $P$?


\item % LB-2016-1
Let $ABCD$ be a parallelogram.
Let $M$, $N$, $O$, and $P$ be the midpoints of $AB$, $BC$, $CD$, and $DA$ respectively.
A circle $\Gamma_1$, centred at $M$, passes through $A$ and $B$; a circle $\Gamma_3$, centred at $O$, passes through $C$ and $D$; a circle $\Gamma_2$, centred at $N$, is tangent to $\Gamma_1$ and to $\Gamma_3$; and a circle $\Gamma_4$, centred at $P$, is tangent to $\Gamma_1$, to $\Gamma_2$, and to $\Gamma_3$.
Find the ratio $AB:BC$.


\item % Vietnam 2019 Q2
Consider a sequence $(x_n)$ of integers such that $0 \leq x_0 < x_1 \leq 100$ and $x_{n+1} = 7x_{n+1} -x_n +280$ for all integers $n \geq 0$.
\begin{enumerate}
	\item Show that if $x_0 = 2$ and $x_1 = 3$, then for all positive integers $n$ the sum of the positive factors of $x_n x_{n+1} +x_{n+1}x_{n+2} +x_{n+2}x_{n+3} +2018$ is divisible by $24$.
	\item Find all pairs $(x_0,x_1)$ such that $x_n x_{n+1} +2019$ is a perfect square for infinitely many values of $n$.
\end{enumerate}


\item % Bulgaria 2017 Q3
Let $M$ be a finite set of positive integers.
For every nonempty $A \subseteq M$ we define
\[ f(A) = \{x \in M \mid x \ \text{is divisible by an odd number of elements of} \ A\}. \]
Find the minimum number of colours needed such that it is possible to assign a colour to each nonempty subset of $M$ in such a way that whenever $A \neq f(A)$, the sets $A$ and $f(A)$ have different colours.

\end{enumerate}


\vfill
\small
\begin{itemize}
	\item Submit your solutions at \href{https://forms.gle/Pv89v957obJMEAw26}{https://forms.gle/Pv89v957obJMEAw26}
	\item Submit each question in a single separate PDF file (with multiple pages if necessary).
	\item If you take photographs of your work, use a document scanner such as Office Lens to convert to PDF.
	\item If you have multiple PDF files for a question, combine them using software such as PDFsam.
\end{itemize}

\vfill
% ASCII art
\centering
\small
\begin{BVerbatim}
 _________         .    .
(..       \_    ,  |\  /|
 \       O  \  /|  \ \/ /
  \______    \/ |   \  / 
     vvvv\    \ |   /  |
     \^^^^  ==   \_/   |
      `\_   ===    \.  |
      / /\_   \ /      |
      |/   \_  \|      /
             \________/
\end{BVerbatim}

\end{document}
