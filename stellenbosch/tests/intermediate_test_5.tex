\documentclass{article}

\usepackage{mathtools,amsfonts}
\usepackage{enumitem}
\usepackage{fullpage}
\usepackage{fancyvrb}
\usepackage{hyperref}


\begin{document}
\thispagestyle{empty}

\begin{center}
  \textbf{\Large Intermediate Test 5}
  \\ \vspace{1em}
  \textbf{\large Stellenbosch Camp 2021}
  \\ \vspace{1em}
  \textbf{\large Time: $4\frac{1}{2}$ hours}
\end{center}

\bigskip

\begin{enumerate}[itemsep=\fill]

\item % Estonia 2006
Let $n$ be a $k$ digit number. $n$ is said to be a \textit{Stellenbosch prime} if every segment of consecutive digits from $n$ (this segment could have $1,2,...$ or $k$ digits) is prime. Find all Stellenbosch primes.


\item %Estonia 2006
Square $ABCD$ has centre $K$. If point $P$ is different from $K$ such that $\angle APB = 90^{\circ}$, prove that $KP$ bisects one of the angles formed by lines $AP$ and $AP$.

\item % 


\item % Estonia 2006
A maths test written by $s$ students is comprised of $q$ questions. A question in the test is called $easy$ if more than half of the students solve it. A student fails the test if they don not solve at least half of the questions. Find all possible pairs $(s, q)$ such that it is possible for:
\begin{enumerate}
\item All the questions to be easy when all the students fail the exam
\item None of the questions to be easy when none of the students fail the exam 
\end{enumerate}

\item % 


\item % 

\end{enumerate}


\vfill
\begin{itemize}
	\item Submit your solutions at \href{https://forms.gle/T9HNgZgj8EhypBnR6}{https://forms.gle/T9HNgZgj8EhypBnR6}
	\item Submit each question in a single separate PDF file (with multiple pages if necessary).
	\item If you take photographs of your work, use a document scanner such as Office Lens to convert to PDF.
	\item If you have multiple PDF files for a question, combine them using software such as PDFsam.
\end{itemize}

\vfill
% ASCII art
\centering
\small
\begin{BVerbatim}
               _,.---.---.---.--.._ 
           _.-' `--.`---.`---'-. _,`--.._
          /`--._ .'.     `.     `,`-.`-._\
         ||   \  `.`---.__`__..-`. ,'`-._/
    _  ,`\ `-._\   \    `.    `_.-`-._,``-.
 ,`   `-_ \/ `-.`--.\    _\_.-'\__.-`-.`-._`.
(_.o> ,--. `._/'--.-`,--`  \_.-'       \`-._ \
 `---'    `._ `---._/__,----`           `-. `-\
           /_, ,  _..-'                    `-._\
           \_, \/ ._(
            \_, \/ ._\
             `._,\/ ._\
               `._// ./`-._
                 `-._-_-_.-
\end{BVerbatim}

\end{document}
