\documentclass[10pt,a4paper]{article}
\usepackage{amsmath}
\usepackage{fancyhdr}
\usepackage{amsthm}

\pagestyle{fancy}
\fancyhf{}
\fancyhead[LE,RO]{Intermediate Combinatorics}
\fancyhead[RE,LO]{Stellenbosch Camp 2021}

\begin{document}

\begin{enumerate}

\item How many ways are there to arrange n people with distinct heights in a line such that everybody is either taller than everyone in front of them or shorter than everyone in front of them?
 \textit{(Canada 1989)} 

\textbf{Solution} There are ${2}^{n-1}$ways of doing this, as we will show by induction. Note that when $n=1$, there is only one way. When there are $n$ people, note that the last person must either be the tallest or shortest of the $n$. In either case, the first $n-1$ places must now be filled in by $n-1$ people with the same restrictions as before, which by the inductive hypothesis can be done in $2^{n-2}$ ways, making the total number of ways $2\times 2^{n-2} = 2^{n-1}$.

\item What is the largest possible size of a subset $S$ of $\{1,2,...,9\}$ if the sums of each pair of distinct elements from $S$ must all be different? 
\textit{(Canada 2002)}

\textbf{Solution} The maximum possible number of elements is $5$. The subset $\{1,2,3,5,8\}$ has 5 elements, and no 2 pairs sum to the same value. To see that no subset $S$ of size 6 exists, consider that $S$ would give $\binom{6}{2} = 15$ different pairs, each of which must sum to a number in the range $3-17$. This range only contains $15$ integers, so each possibility must be used exactly once. Hence the elements of the only pair summing to $3$ (i.e.~$1,2$) and the only pair summing to $17$ (i.e.~$8,9$) must all be in $S$. But then $1+9=10=2+8$, so 2 pairs in $S$ have the same sum, a contradiction.

\item Aaron attempts all six questions on the SAMO senior paper. For each question, his mark is an integer from 0 to 7. He never scores more points on a later question than on any earlier question. How many different possible sequences of six marks can he achieve? 
\textit{(BMO1 2009/10)}

\textbf{Solution} Let $x_i$ be the number of questions for which Aaron scores $i$ points. ($i\in\{0,1,2,...,7\}$). Note that once the $x_i$ have been found, the questions each score corresponds to is uniquely determined by the condition that higher scores must come first. So we need to count the nonnegative integer solutions to the equation: $x_0 + x_1 + x_2 + ... + x_7 = 6$ ,which is done using the "stars and bars" technique. In this case, we have $6$ stars and $7$ bars ($8$ $x_i's$), and the answer is $\binom{13}{6}$ (or $1716$).

\item A classroom has desks arranged in two rows of ten desks. How many ways are there to choose seven of these desks such that no two chosen desks in the same row are adjacent?
\textit{(BMO1 2014/15)}

\textbf{Solution} 

\textit{Lemma:} The number of ways to choose $c$ desks in one row such that no two are adjacent is $\binom{11-c}{c}$

\textit{Proof:} Suppose the desks chosen are numbered $x_1, x_2, ..., x_c$. ($1\leq x_i \leq 10$ and $x_{i+1}-x_i \geq 2$). Then there is a bijection between such sequences and sequences $y_1, y_2, ... y_c$  ($0\leq y_i\leq10-c$ and $y_{i+1}-y_i \geq1$) realised by the transformation $y_i = x_i - i$. Hence the number of ways to choose the desks equals the number of ways to choose $c$ numbers to be $y_i's$ out of the $11-c$ possibilities. \qed

We can choose at most $5$ desks in each row, meaning we can choose $2,3,4$ or $5$ in the front row and $5,4,3$ or $2$ in the back row respectively. Using the lemma, this means our answer is $$\binom{9}{2}\times\binom{6}{5} +  \binom{8}{3}\times\binom{7}{4} + \binom{7}{4}\times\binom{8}{3} + \binom{6}{5}\times\binom{9}{2} = 4352$$

\item A group of $2021$ students write a test that consists of $2021$ questions. Each question is solved by at least $1011$ students. Prove that there is a group of 10 students such that each question was solved by at least one member of the group.
\textit{(Portugal 2010)}

\textbf{Solution}

\textit{Lemma:} Given any subset of $x$ questions, there is a student that solved more than $\frac{x}{2}$ of them. 

\textit{Proof:} Each of the $x$ question is solved by at least 1011 students, so there at least $1011x$ scripts for these questions that will be marked correct. By the pigeonhole principle, there is a student that solves $\geq \lceil\frac{1011x}{2021}\rceil > \frac{x}{2}$ questions. \qed

Hence there is a student that solved at least 1011 questions. Of the remaining 1010 questions, there is a (not neccessarily different) student that solved at least 506. Continuing in this way, we get 10 students who solved $1011,506,253,126,63,32,16,8,4$ and $2$ questions respectively. (If any two of these happen to be the same student, we can choose any other student at random to fill the second space.)

\item Every contestant in an arm-wrestling competition wrestles once with each of the others. If no match ends in a draw and nobody loses all of their matches, prove that there must be three contestants $X,Y,Z$ such that in their matches, $X$ defeated $Y$, $Y$ defeated $Z$ and $Z$ defeated $X$.  
\textit{(Ireland 2004)}

\textbf{Solution:} Let $X$ be the contestant who won the fewest matches (if there is a tie, choose one arbitrarily). $X$ must have won at least one match, let his opponent for that match be $Y$. Suppose $Y$ won exactly $n$ matches, against contestants $Z_1,Z_2,...,Z_n$. If $X$ beat each $Z_i$ for $0\leq i\leq n$ then, along with his win against $Y$, $X$ would have won at least $n+1$ matches, more than $Y$, which is impossible since $X$ won the fewest number of matches. So for some $Z_i$, $Z_i$ beat $X$. Along with the facts that $X$ beat $Y$ and $Y$ beat $Z_i$, we have our required three contestants.

\item How many ways are there to choose 24 black squares from a standard chess board (with 32 black and 32 white squares alternating) such that exactly three squares are chosen in each row and in each column?
\textit{(Ireland 2014)}

\textbf{Solution:} Since each row and each solumn has 4 black squares, the question is equivalent to choosing 8 black squares such that each row and each column has one chosen square. Considering rows 1,3,5 and 7, we see there there are 4 options to choose the square in row 1, then 3 options in row 3, 2 options in row 5, and 1 in row 7, for a total of $4! = 24$ options. Similarly, there are 24 options to choose the squares in rows 2,4,6 and 8. This gives a total of $24\times24 = 576$ possibilities.

\item Phil and Phyllis play a game using a map containing $n$ cities. Initially, Phil chooses two cities $S$ and $F$, and places a figurine on $S$. Phyllis then draws $m$ roads between pairs of cities on the map. At most one road is drawn between any pair of cities, and no road is drawn from a city to itself. Thereafter, on Phil's move he may move the figurine from its current city along any road to a new city. On her move, Phyllis may remove any road from the map. Phil wins the game if he can get the figurine to city $F$. If moves alternate with Phil moving first, what is the smallest $m$ needed to give Phil a winning strategy?  
\textit{(Junior Malaysia Olympiad, 2015)}

\textbf{Solution:} The smallest such $m$ is $m=\binom{n}{2} = \frac{n(n-1)}{2}$. In this case, there is a road between every two cities, so Phil moves the figurine from $S$ to $F$ on his first move and wins. If $m = \frac{n(n-1)}{2}-1$, then Phyllis wins as follows: Initially, draw roads between every pair of cities except $S$ and $F$. Then, on Phil's turn, he cannot move the token to $F$. Whenever Phil moves the counter to a city $C$ on his move, Phyllis removes the road between $C$ and $F$ (if it exists, otherwise any other road). This way, Phil never has the opportunity to move to $F$ and so cannot win. 
\end{enumerate}


\end{document}