\documentclass{article}

\usepackage{mathtools,amsfonts}
\usepackage{enumitem}
\usepackage{fullpage}
\usepackage{fancyvrb}
\usepackage{hyperref}


\begin{document}
\thispagestyle{empty}

\begin{center}
  \textbf{\Large Advanced Test 5}
  \\ \vspace{1em}
  \textbf{\large Stellenbosch Camp 2021}
  \\ \vspace{1em}
  \textbf{\large Time: $4\frac{1}{2}$ hours}
\end{center}

\bigskip

\begin{enumerate}[itemsep=\fill]

\item % 2012 Sharygin Geometry Olympiad, Final Round, Grade 8
Let $ABC$ be a triangle with $AB=AC$, and let $D$ be the midpoint of $BC$. Let $E$ be the reflection of $D$ across $AC$. Let $F$ be the point on $AB$ such that $FE||BC$. Prove that $AB\perp FC$.


\item % Number Theory Marathon Notes, Problem 13
Let $A$ be a natural number with $2^n$ digits, all of which are equal. Show that $A$ has at least $n$ distinct prime factors.


\item %
Let $A$ and $B$ be two points in the plane with distance $1$ between them.
Find the maximum length of a path from $A$ to $B$, comprising at most $n$ straight line segments, with the property that at every point along the path (not only at the endpoints of the segments) as one moves from $A$ to $B$ the distance to $B$ is reducing.


\item % St. Petersburg 1996
At an $n+1$ day long maths camp with $n$ students, some pairs of students start the camp as friends with each other. On the $i^{th}$ evening ($1 \leq i \leq n$), the $i^{th}$ student hosts a mafia game and invites all their friends at the camp (whether they started the camp as friends or became friends on an earlier evening). At any mafia game, participants have such a good time that each pair of students at the game become friends with each other. On evening $n+1$, one of the $n$ students in the camp hosts one final mafia game and invites all their friends. Prove that no new friendships are formed in this final mafia game.


\item % 
Let $ABC$ be a triangle with circumcircle $\Gamma$. The line tangent to $\Gamma$ at $A$ meets $BC$ at $D$. Let the circumcircle of $\triangle ACD$ be $\Omega$. The tangents to $\Omega$ at $A$ and $C$ meet at $T$. The lines $DT$ and $AB$ meet at $M$. Prove that $M$ is the midpoint of $AB$.


\item % 

\end{enumerate}


\vfill
\small
\begin{itemize}
	\item Submit your solutions at \href{https://forms.gle/T9HNgZgj8EhypBnR6}{https://forms.gle/T9HNgZgj8EhypBnR6}
	\item Submit each question in a single separate PDF file (with multiple pages if necessary).
	\item If you take photographs of your work, use a document scanner such as Office Lens to convert to PDF.
	\item If you have multiple PDF files for a question, combine them using software such as PDFsam.
\end{itemize}

\vfill
% ASCII art
\centering
\small
\begin{BVerbatim}
                     _ _ _
                ,="`/ / /'=.
               / / / / / / /'.
              /_/_/_/_/_/ / / \     _.='/
           .-' - _ - _ -`"-,/ /\  .'_.='/
          / -_ - _ - _ - _ -\/.'|/_.=`/
         / @    _="`} _- _- _\.=/_. =";
        /     -"_="} - _  - _ -=_-_"=;
        \-.   '=._}          ,._--_=_;
         `-._     ._        /;' \ `"=.;
             `"\`;-.}_ _ _.;\ \/ \'=._\
                \ =.}\ \ \ \ \'   '._=_.\
                 \_}`=._\_\.'`       '=.\
\end{BVerbatim}

\end{document}
