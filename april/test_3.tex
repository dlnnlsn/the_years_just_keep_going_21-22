\documentclass{article}

\usepackage{mathtools,amsfonts}
\usepackage{enumitem}
\usepackage{fullpage}
\usepackage{fancyvrb}


\begin{document}
\thispagestyle{empty}

\begin{center}
  \textbf{\Large Test 3}
  \\ \vspace{1em}
  \textbf{\large Stellies March Camp 2022}
  \\ \vspace{1em}
  \textbf{\large Time: $4\frac{1}{2}$ hours}
\end{center}

\vspace{24pt}

\begin{enumerate}[itemsep=12pt]

\item %Balkan MO 2014
A \textit{special number} is a positive integer $n$ for which there exists positive integers $a$, $b$, $c$ and $d$ with $$n = \frac{a^3+2b^3}{c^3+2d^3}$$
Prove that:
\begin{enumerate}
	\item there are infinitely many special numbers;
	\item $1919$ is \textit{not} a special number.
\end{enumerate}


\item % C2
Let $n \geq 3$ be an integer.
An integer $m \geq n+1$ is called \emph{$n$-colourful} if, given infinitely many marbles in each of $n$ colours, it is possible to place $m$ of them around a circle so that in any group of $n+1$ consecutive marbles there is at least one marble of each colour.

Prove that there are only finitely many integers which are not $n$-colourful, and find the largest among them.

\item % A3
\newcommand{\floorf}[2]{\left\lfloor\frac{#1}{#2}\right\rfloor}
Given a positive integer $n$, find the smallest value of
\[ \floorf{a_1}{1} +\floorf{a_2}{2} +\dotsb +\floorf{a_n}{n} \]
over all permutations $(a_1, a_2, \dotsc, a_n)$ of $(1, 2, \dotsc, n)$.

\end{enumerate}

\vfill
% ASCII art
\centering \small
\begin{BVerbatim}
                          '
                        '   '
                      '       '
                 .  '  .        '                        '
             '             '      '                   '   '
          '                    '  . '              '      '
       '                             .          '        '
    '                                   '  . '         '
  '                                                   .
.    ()     .                                        .
 .                                                    '
   .        '  .'''.                    . . .           .
      .    '   '....'               ..'.      ' .
         '  .                     .     '          '     '
               '  .  .  .  .  . '.    .'              '  .
                   '         '    '. '
                     '       '      '
                       ' .  '
                          '
\end{BVerbatim}

\end{document}
