\documentclass{article}

\usepackage{mathtools,amsfonts}
\usepackage{enumitem}
% \usepackage{fullpage}
\usepackage{fancyvrb}


\begin{document}
\thispagestyle{empty}

\begin{center}
  \textbf{\Large Test 2}
  \\ \vspace{1em}
  \textbf{\large Stellies March Camp 2022}
  \\ \vspace{1em}
  \textbf{\large Time: $4\frac{1}{2}$ hours}
\end{center}

\vspace{24pt}

\begin{enumerate}[itemsep=12pt]

\item %
Let $ABC$ be an acute triangle.
Let $H$ denote the orthocentre and $D$, $E$ and $F$ the feet of its altitudes from $A$, $B$ and $C$, respectively.
Let the common point if $DF$ and the altitude through $B$ be $P$.
The line perpendicular to $BC$ through $P$ intersects $AB$ in $Q$.
Furthermore, $EQ$ intersects the altitude through $A$ in $N$.
Prove that $N$ is the midpoint of $AH$.

\item % N3
Find all positive integers $n$ with the following property:
the $k$ positive divisors of $n$ have a permutation $(d_1, d_2, \dotsc, d_k)$ such that for every $i = 1, 2, \dotsc, k$ the number $d_1 +d_2 +d\dotsb +d_i$ is a perfect square.

\item % C5
Let $n$ and $k$ be two integers with $n > k \geq 1$.
There are $2n+1$ students standing in a circle.
Each student $S$ has $2k$ \emph{neighbours}---namely, the $k$ students closest to $S$ on the right and the $k$ students closest to $S$ on the left.

Suppose that $n+1$ of the students are girls, and that the other $n$ are boys.
Prove that there is a girl with at least $k$ girls among her neighbours.

\end{enumerate}

\vfill
% ASCII art
\centering
\begin{BVerbatim}
\end{BVerbatim}

\end{document}
